% !TeX root = ../easyfloats.tex

% Copyright © 2020 E. Zöllner
% Alternatively to the terms of the LPPL, at your choice,
% you can redistribute and/or modify this file under the
% terms of the Do What The Fuck You Want To Public License, Version 2,
% as published by Sam Hocevar. See http://www.wtfpl.net/about/.

\usepackage{etoolbox}
\usepackage{pgfkeys}

% Defines a macro to document a macro or environment key:
%     \keydoc{<keyname> = <allowedvalues>}{<keytype>}
%     \keydoc[<options>]{<keyname> = <allowedvalues>}{<keytype>}
% Following text is assumed to be a description of the key.
% If it is instead followed by another \keydoc it is assumed
% that this key has no description.
% If a description applies to two keys separate the \keydocs
% by a slash.
% Do *not* put an empty line between a \keydoc and it's description.
% Spaces after (the second argument of) \keydoc are ignored.
% <options> is a comma separated list of the following possible key=value pairs:
% - target = the full key to another key.
%   The name of this key is made into a link to target.
%   This is used by \forwardingkeydoc.
% - initial value
% - default value
% Values are separated by pipes.
% Wildcards are put in angular brackets.
%
% The description of a \keydoc is assumed to reach up to the
% next \keydoc, \section, \subsection or \subsubsection.
% If that is not the case you must explicitly end it with
%     \endkeydoc
% If the description of a \keydoc is ended by a \*section macro
% there must be a \par (aka an empty line) before that \*section.
% Otherwise \keydoc@descriptionsep will be inserted.
%
% There is a special macro for forwarding keys:
%     \forwardingkeydoc[<options>]{<keyname>}
% In contrast to \keydoc, the target option can be a path, name or full key.
% If target is not given it is tried to generate it by expanding
%     \TargetKey{<keypath>}{<keyname>}
% This macro is given the path (without trailing slash) and name
% of the key to be described and is expected to expand to a full key.
% You must define it yourself if you want to leave out the optional argument.
%
% \forwardingkeydoc uses internally a more generalized macro:
%     \correspondingkeydoc[<options>]{<keyname>}{<keytype>}
% target is as flexible as for \forwardingkeydoc
% but it also takes a <keytype> like \keydoc.
%
% Examples for a key pattern can be specified with
%     \keylinktarget{<keyexample>}
% It must be specified directly following the \keydoc it refers to.
%
% \key is redefined to link to \keydoc.
% However, different environments can have different keys
% with the same name. In that case a path must be set to
% distinguish these keys:
%     \keydocpath{<path>}
% Sets the path for all following \keydoc in the same group.
% Unambiguous keys can be referenced without a path:
%     \key{sub}
% For ambiguous keys the path must given as an optional argument:
%     \key[/object]{label}
% Paths may *not* include a trailing slash.
%
% Internals:
% \begin@keydoc is put before the first \keydoc.
% \endkeydoc is put after the last description of a \keydoc.
% \keydoc@beforekey is put before each \keydoc but only once
%                   for \keydocs which belong together.
% \keydoc@keysep is put between \keydocs which belong together.
% \keydoc@descriptionsep is put between a \keydoc and it's description.
% If a \keydoc is followed by a \par neither \keydoc@keysep
% nor \keydoc@descriptionsep is inserted.


\makeatletter

% ---------- \keydoc settings ---------

\newcommand{\begin@keydoc}{\begin{itemize}}
\def        \endkeydoc    {\end{itemize}}

\newcommand{\keydoc@beforekey}{\item}
\newcommand{\keydoc@keysep}{~/ }
\newcommand{\keydoc@descriptionsep}{\nopagebreak\par}

\newcommand{\keydoc@mandatoryvalueformat}[1]{#1}
\newcommand{\keydoc@optionalvalueformat}{\textcolor{gray}}

\newcommand{\keydoc@keytype}[1]{\,\ensuremath{^{\pgfmanualcslinkpreskip=.33em\pgfmanualcslinkpostskip=\pgfmanualcslinkpreskip\LinkUnformattedKeyword{keytype}{#1}{\keytypesymbol{#1}}}}}
\newcommand{\keytypedoc@keytype}[1]{\raisebox{.2ex}{\bfseries\keytypesymbol{#1}}}

% ---------- key type symbols ---------

\newcommand{\keytypesymbolformat}[1]{\hbox{\scriptsize(#1)}}
\newcommand{\newkeytypesymbol}[2]{% #1: key type, #2: symbol
	\expandafter\newcommand\csname keytypesymbol@#1\endcsname{#2}%
}
\newcommand{\keytypesymbol}[1]{% #1: key type
	\@ifundefined{keytypesymbol@#1}%
		{\PackageError{keydoc}{undefined keytype `#1'}{}}%
		{\keytypesymbolformat{\csname keytypesymbol@#1\endcsname}}%
}

\newkeytypesymbol{storing key}{sto}
\newkeytypesymbol{executed key}{exe}
\newkeytypesymbol{boolean key}{bool}
\newkeytypesymbol{forwarding key}{fwd}
\newkeytypesymbol{handler}{hdl}
\newkeytypesymbol{unknown key handler}{unk}


% ---------- optional dependencies ---------

\providecommand{\key}[2][]{% #1: key path/group, #2: key name
	\texttt{#2}%
}


% ---------- auxiliary macros ---------

% \ifendsonspace{<text>}{<then>}{<else>}
% Checks if the last token of <text> is a space.
% If it is it expands to <then>{<text-without-last-space>}.
% If it is not it expands to <else>{<text>}.
% <text> may not contain \relax.
\newcommand\ifendsonspace{\ifendson{ }}
\newcommand\ifendson[2]{%
	\def\endson@check##1#1\relax.##2\relax\relax##3##4{%
		\if \relax\detokenize{##2}\relax
			\def\endson@do{##4}%
			\endson@stripandappto##1%
		\else
			\def\endson@do{##3{##1}}%
		\fi
		\endson@do
	}%
	\def\endson@stripandappto##1\relax.{\appto\endson@do{{##1}}}%
	\endson@check#2\relax.#1\relax.\relax\relax
}

% \newstripoptend{<cs>}{<suffix>}
% Defines a new control sequence <cs> which checks if it's argument ends on <suffix>.
% If the argument of <cs> ends on <suffix> it is stripped.
% If the argument of <cs> does not end on <suffix> <cs> expands to the argument.
% <cs> is expandable.
% The number of required expansion steps depends on whether the argument ends on <suffix> or not.
\newcommand\newstripoptend[2]{% #1: cs, #2: end
	\expandafter\def\csname\string#1@do\endcsname##1#2\relax.##2\relax\relax{%
		\if \relax\detokenize{##2}\relax
			\expandafter\@firstoftwo
		\else
			\expandafter\@secondoftwo
		\fi
			{\csname\string#1@strip\endcsname##1}%
			{##1}%
	}%
	\expandafter\def\csname\string#1@strip\endcsname##1\relax.{##1}%
	\newcommand#1[1]{\csname\string#1@do\endcsname##1\relax.#2\relax.\relax\relax}%
}
\newstripoptend\stripoptplus{ (+)}


% ---------- \FormatKey ---------

\newcommand{\FormatKey}[1]{%
	\def\FormatKeyText{}%
	\FormatKey@lookahead#1\pgfeov
	\let\FormatKey@original=\FormatKey
	\let\FormatKey=\@firstofone
	\KeyFormat@original@key{\FormatKeyText}%
	\let\FormatKey=\FormatKey@original
}
\newcommand{\FormatKey@lookahead}{%
	\futurelet \FormatKey@nexttoken \FormatKey@check
}
\newcommand{\FormatKey@check}{%
	\ifx \FormatKey@nexttoken (%
		\def\FormatKey@do{\FormatKey@optional}%
	\else\ifx \FormatKey@nexttoken <%
		\def\FormatKey@do{\FormatKey@var}%
	\else\ifcat \noexpand\FormatKey@nexttoken \space
		\def\FormatKey@do{\FormatKey@space}%
	\else\ifx \FormatKey@nexttoken \pgfeov
		\def\FormatKey@do{\FormatKey@endofloop}%
	\else\ifx \FormatKey@nexttoken \bgroup
		\def\FormatKey@do{\FormatKey@arg}%
	\else
		\def\FormatKey@do{\FormatKey@normaltoken}%
	\fi \fi \fi \fi \fi
	\FormatKey@do
}
\def\FormatKey@normaltoken#1{%
	\appto\FormatKeyText{#1}%
	\FormatKey@lookahead
}
\def\FormatKey@arg#1{%
	\let\FormatKey@text@backup=\FormatKeyText
	\def\FormatKeyText{}%
	\FormatKey@lookahead#1\pgfeov
	\eappto\FormatKey@text@backup{{\expandonce{\FormatKeyText}}}%
	\let\FormatKeyText=\FormatKey@text@backup
	\FormatKey@lookahead
}
\def\FormatKey@optional(#1){%
	\appto\FormatKeyText{{\normalfont\color{gray}(}}%
	\FormatKey@lookahead#1\pgfeov
	\appto\FormatKeyText{{\normalfont\color{gray})}}%
	\FormatKey@lookahead
}
\providecommand{\meta@nolink}{\meta}
\def\FormatKey@var<#1>{%
	\appto\FormatKeyText{{\normalfont\meta@nolink{#1}}}%
	\FormatKey@lookahead
}
\@firstofone{\def\FormatKey@space} {%
	\appto\FormatKeyText{ }%
	\FormatKey@lookahead
}
\def\FormatKey@endofloop\pgfeov{%
}

\providecommand{\pgfeov}{\textcolor{red}{\textbf{eov, this should never be expanded}}}

\let\KeyFormat@original@key=\key
\renewcommand{\key}{\FormatKey}


% ---------- format allowed values ---------

\def\FormatVal@var#1{{\normalfont\meta{#1}}}
\def\FormatVal@or{\unskip\,\ensuremath{|}\,\ignorespaces}

\newcommand{\FormatValue}[1]{%
	\def\FormatValue@val{}%
	\expandafter\FormatValue@loop#1\pgfeov
	\FormatValue@val
}
\newcommand{\FormatValue@loop}{%
	\futurelet \FormatValue@nexttoken \FormatValue@check
}
\newcommand{\FormatValue@check}{%
	\ifx \FormatValue@nexttoken \pgfeov
		\def\FormatValue@do##1{}%
	\else\ifcat \noexpand\FormatValue@nexttoken \space
		\appto\FormatValue@val{ }%
		\def\FormatValue@do##1{\FormatValue@loop##1}%
	\else
		\def\FormatValue@do{\FormatValue@check@ii}%
	\fi \fi
	\FormatValue@do
}
\newcommand{\FormatValue@check@ii}[1]{%
	\if|\noexpand#1%
		\appto\FormatValue@val{\FormatVal@or}%
		\def\FormatValue@do{\FormatValue@loop}%
	\else\if<\noexpand#1%
		\def\FormatValue@do##1>{\appto\FormatValue@val{\FormatVal@var{##1}}\FormatValue@loop}%
	\else
		\appto\FormatValue@val{#1}%
		\def\FormatValue@do{\FormatValue@loop}%
	\fi \fi
	\FormatValue@do
}


% ---------- keyword ---------
% This is a generalized api to link certain keywords.
% Use the following macro in a description:
%     \NewKeyword[\unexpanded]{<keywordtype>}{<keyword>}{<formatmacro-bg>}{<formatmacro-fg>}{<text>}
% Use the following macro in references:
%     \LinkKeyword{<keywordtype>}{<keyword>}<formatmacro-bg><formatmacro-fg>{<text>}
% <keywordtype> and <keyword> are used to generate/
% link to a hyper link target. They are not visible
% in the resulting document.
% <formatmacro-bg/fg>: TeX code which takes one mandatory argument
% and formats it as desired. Usually those passed to \NewKeyword are used
% for both the description and the link.
% Those passed to \LinkKeyword are used only if the link is undefined.
% <formatmacro-bg> is applied before \FormatLink.
% <formatmacro-fg> is applied inside of \FormatLink.
% Setting a background color must happen in <formatmacro-bg>
% otherwise it will cover the underline produced by \FormatLink.
% Using \detokenize must happen in <formatmacro-fg>
% otherwise \FormatLink will not be executed but printed to the pdf.
% Changing the font can happen in either one.
% <text>: TeX code to produce the displayed text.
% The optional argument of \NewKeyword is a macro which
% takes the <keyword> as argument and can protect it
% from expansion when writing it to an auxiliary file.
% If you try to link an undefined keyword you will
% get a warning.
%
% You can define abbreviations so you do not always
% need to type out the full name:
%     \AbbrevKeyword[\unexpanded]{<abbrevtype>}{<abbrevname>}{<fulltype>}{<fullname>}
% If two abbreviations are ambiguous they will not
% cause an error where they are defined but where
% they are referenced so you can create abbreviations
% automatically without trouble.
%
% In order to avoid warnings for undescribed keywords
%     \NoKeyword[\unexpanded]{<keywordtype>}{<keyword>}{<formatmacro-bg>}{<formatmacro-fg>}
% can be used. It is used similar to \NewKeyword but
% does not display any text and \LinkKeyword to this keyword
% will not result in a link and will not give a warning.
% The separation in <formatmacro-bg> and <formatmacro-fg>
% is only for compatability with \NewKeyword.
%
% Links are defined in a temporary file which is loaded in the
% preamble so that \LinkKeyword can be used before \NewKeyword.
% I am not using the aux file for that because
% (a) the aux file is loaded twice (once in \begin{document}
% and once in \end{document}) and
% (b) the aux file is loaded in a group so all definitions need
% to be global but \newcommand can't do that and \gdef does not
% check if a macro is already defined.


% https://github.com/pgf-tikz/pgf/blob/master/tex/latex/pgf/doc/pgfmanual-en-macros.tex
\colorlet{linkcolor}{black!8}
\newdimen\pgfmanualcslinkpreskip
\newdimen\pgfmanualcslinkpostskip
\ifunderlinelinks
\newcommand{\FormatLink}[1]{%
	\begingroup
	\setbox0=\hbox{#1}%
	\rlap{{%
		\keyword@original@color{linkcolor}%
		\dimen0\wd0
		\advance\dimen0by-\pgfmanualcslinkpreskip
		\advance\dimen0by-\pgfmanualcslinkpostskip
		\hskip\pgfmanualcslinkpreskip
		\vrule width\dimen0 height-1pt depth1.6pt
	}}%
	\box0%
	\endgroup
}
\else
\newcommand{\FormatLink}[1]{#1}
\fi


% is used in the first run when the keyword is not yet defined
% or if an abbreviation is ambiguous
\newcommand{\keyword@placeholder}[1]{%
	{\keyword@original@color{yellow}%
		#1%
	}%
}
\let\keyword@original@color=\color

\newcommand{\keyword@targetname}[2]{% #1: keyword type, #2: name
	keyword:#1:#2%
}
\newcommand{\keyword@csname}[2]{% #1: keyword type, #2: name
	keyword@#1@#2%
}
\newcommand{\keyword@link}[3]{% #1: keyword type, #2: name, #3: text
	\hyperlink{\keyword@targetname{#1}{#2}}{#3}%
}
\newcommand{\DisableLinks}{%
	\let\keyword@link=\@thirdofthree
	\let\FormatLink=\@firstofone
}
\let\keyword@formatenable=\@firstofone
\newrobustcmd{\keyword@new}[4]{% #1: keyword type, #2: name, #3: format macro bg, #4: format macro fg
	\expandafter\newcommand\csname\keyword@csname{#1}{#2}\endcsname[1]{\keyword@link{#1}{#2}{\keyword@formatenable{#3}{\FormatLink{\keyword@formatenable{#4}{##1}}}}}%
}
\newrobustcmd{\keyword@no}[4]{% #1: keyword type, #2: name, #3: format macro bg, #4: format macro fg
	\expandafter\newcommand\csname\keyword@csname{#1}{#2}\endcsname[1]{\keyword@formatenable{#3}{\keyword@formatenable{#4}{##1}}}%
}
\newrobustcmd{\keyword@abbrev}[4]{% #1: abbrev keyword type, #2: abbrev name, #3: full keyword type, #4: full name
	\IfKeywordUndefined{#3}{#4}{%
		\PackageError{keyword}{#3 `#4' (for which you wanted to add the abbreviation `#1@#2') is undefined}{}%
	}{\IfKeywordUndefined{#1}{#2}{%
		\expandafter\let \csname\keyword@csname{#1}{#2}\expandafter\endcsname \csname\keyword@csname{#3}{#4}\endcsname
	}{%
		\expandafter\def \csname\keyword@csname{#1}{#2}\endcsname ##1{\PackageWarning{keyword}{The abbreviation #1 `#2' is ambiguous. Please specify the full name for ##1}{}\keyword@placeholder{##1}}%
	}}%
}
\newcommand{\IfKeywordIsAbbrev}[4]{% key to be tested, reference key
	\expandafter\ifx \csname\keyword@csname{#1}{#2}\expandafter\endcsname \csname\keyword@csname{#3}{#4}\endcsname
		\expandafter \@firstoftwo
	\else
		\expandafter \@secondoftwo
	\fi
}

\newcommand{\LinkKeyword}[5]{% #1: keyword type, #2: name, #3: formatmacro bg, #4: formatmacro fg, #5: displayed text
	\IfKeywordUndefined{#1}{#2}{%
		\edef\do{\noexpand\PackageWarning{keydoc}{Either #1 `#2' is undefined or you need to rerun LaTeX}}%
		\do
		\keyword@placeholder{#3{#4{#5}}}%
	}{%
		\csname \keyword@csname{#1}{#2}\endcsname{#5}%
	}%
}
\newcommand{\LinkUnformattedKeyword}[3]{% #1: keyword type, #2: name, #3 displayed text
	\begingroup
		\let\keyword@formatenable=\@gobble
		\LinkKeyword{#1}{#2}\@firstofone\@firstofone{#3}%
	\endgroup
}
\newcommand{\IfKeywordUndefined}[2]{% #1: keyword type, #2: name, #3: then, #4: else
	\expandafter \ifx \csname\keyword@csname{#1}{#2}\endcsname \relax
		\expandafter\@firstoftwo
	\else
		\expandafter\@secondoftwo
	\fi
}


\newcommand\keyword@filename{\jobname.kw}

\AtBeginDocument{%
	\IfFileExists{\keyword@filename}{%
		\makeatletter
		\input{\keyword@filename}%
		\makeatother
	}{}
	\newwrite\keyword@file
	\immediate\openout\keyword@file=\keyword@filename\relax
}

\newcommand{\NewKeyword}[6][\@firstofone]{% [#1: name protection], #2: type, #3: name, #4: format macro bg, #5: format macro fg, #6: text
	%https://tex.stackexchange.com/a/17138
	\Hy@raisedlink{\hypertarget{\keyword@targetname{#2}{#3}}{}}%
	\if\relax\detokenize{#6}\relax
	\else
		#4{#5{#6}}%
	\fi
	\immediate\write\keyword@file{\keyword@new{#2}{#1{#3}}{\unexpanded{#4}}{\unexpanded{#5}}}%
}
\newcommand{\AbbrevKeyword}[5][\@firstofone]{% abbrev = full
	\immediate\write\keyword@file{\keyword@abbrev{#2}{#1{#3}} {#1{#4}}{#1{#5}}}%
}
\newrobustcmd{\NoKeyword}[5][\@firstofone]{% #2: keyword type, #3: name, #4: format macro bg, #5: format macro fg
	\immediate\write\keyword@file{\keyword@no{#2}{#1{#3}}{\unexpanded{#4}}{\unexpanded{#5}}}%
}


% ---------- \keydoc ---------

\newif\if@inkeydoc
\newif\ifkeydoc@isrelated

\newcommand\keydoc@keypath{}
\newcommand{\keydocpath}[1]{\def\keydoc@keypath{#1}}

\pgfqkeys{/keydoc}{%
	target/.initial,
	target/.value required,
	%
	initial value/.initial,
	initial value/.value required,
	%
	default value/.initial,
	default value/.value required,
}

\newcommand{\keydoc@printvalues}{%
	\pgfkeysgetvalue{/keydoc/initial value}{\keydoc@initialvalue}%
	\pgfkeysgetvalue{/keydoc/default value}{\keydoc@defaultvalue}%
	\let\keydoc@initdefault=\@empty
	%
	\unless\ifx \keydoc@defaultvalue \pgfkeysnovalue@text
		\appto\keydoc@initdefault{\hyperref[initial-vs-default-values]{默认值}: \keydoc@defaultvalue.}%
	\fi
	\unless\ifx \keydoc@initialvalue \pgfkeysnovalue@text
		\unless\ifx \keydoc@initdefault \@empty
			\appto\keydoc@initdefault{ }%
		\fi
		\appto\keydoc@initdefault{\hyperref[initial-vs-default-values]{初值}: \keydoc@initialvalue.}%
	\fi
	\unless\ifx \keydoc@initdefault \@empty
		\\%
		{\MaxFontSize\footnotesize\keydoc@initdefault}%
	\fi
	\pgfkeyssetvalue{/keydoc/initial value}{\pgfkeysnovalue}%
	\pgfkeyssetvalue{/keydoc/default value}{\pgfkeysnovalue}%
}
\newcommand{\MaxFontSize}[1]{%
	#1%
	\def\MaxFontSize@do##1#1##2\relax{%
		\@tfor\MaxFontSize@fontsize:=##1\do{%
			\expandafter\let\MaxFontSize@fontsize=\relax
		}%
	}%
	\MaxFontSize@do\Huge\huge\LARGE\Large\large\normalsize\small\footnotesize\scriptsize\tiny\relax
}

\newcommand{\keydoc}[3][]{%  [#1: options], #2: key name, #3: key type
	\unless \if@inkeydoc
		\begin@keydoc
		\keydoc@registerend
		\@inkeydoctrue
	\fi
	\pgfkeyssetvalue{/keydoc/target}{\pgfkeysnovalue}%
	\pgfqkeys{/keydoc}{#1}%
	\keydoc@parsekey#2=\relax%
	\def\keydoc@abbrev{%
		\AbbrevKeyword{key}{\keydoc@keypattern} {key}{\keydoc@keypath/\keydoc@keypattern}%
		\AbbrevKeyword{}{\keydoc@keypattern} {key}{\keydoc@keypath/\keydoc@keypattern}%
	}%
	\pgfkeysgetvalue{/keydoc/target}{\keydoc@targetkey}%
	\ifx \keydoc@targetkey \pgfkeysnovalue@text
		\let\keydoc@link=\@firstofone
	\else
		\def\keydoc@link{\LinkKeyword{key}{\keydoc@targetkey}\keydoc@key@bg\keydoc@original@key}%
		\IfKeywordUndefined{}{\keydoc@keypattern}{%
			% keep \keydoc@abbrev
		}{\IfKeywordIsAbbrev{}{\keydoc@keypattern} {key}{\keydoc@keypath/\keydoc@keypattern}{%
			% keep \keydoc@abbrev
		}{%
			\let\keydoc@abbrev=\relax
		}}%
	\fi
	\ifkeydoc@isrelated
		\keydoc@keysep
	\else
		\keydoc@beforekey
	\fi
	\edef\keydoc@do{%
		\noexpand\NewKeyword{key}{\keydoc@keypath/\keydoc@keypattern}{\noexpand\keydoc@key@bg}{\noexpand\keydoc@original@key}{\noexpand\keydoc@link{\keydoc@keypattern}%
			\noexpand\keydoc@valueformat{\keydoc@keyvalsep\noexpand\FormatValue\noexpand\keydoc@keyval}}%
	}%
	\keydoc@do
	\keydoc@abbrev
	\keydoc@keytype{#3}%
	\keydoc@isrelatedfalse
	\keydoc@lookahead
}
\def\keydoc@parsekey#1=#2\relax{%
	\if \relax\detokenize{#2}\relax
		\def\keydoc@keypattern{#1}%
		\def\keydoc@keyval{}%
		\def\keydoc@keyvalsep{}%
		\let\keydoc@valueformat=\@firstofone
	\else
		\expandafter\def\expandafter\keydoc@keyval\expandafter{\keydoc@stripeq#2\relax}%
		\expandafter\keydoc@checkifvalueisoptional\keydoc@keyval\relax
		\ifendsonspace{#1}{%
			\def\keydoc@keyvalsep{\ =}%
			\def\keydoc@keypattern%{#1 without space}
		}{%
			\def\keydoc@keyvalsep{=}%
			\def\keydoc@keypattern%{#1}
		}%
	\fi
}
\def\keydoc@stripeq#1=\relax{#1}
\def\keydoc@checkifvalueisoptional#1#2\relax{%
	\pgfkeysgetvalue{/keydoc/default value}{\keydoc@defaultvalue}%
	\ifx?#1%
		\ifx \keydoc@defaultvalue \pgfkeysnovalue@text
			\ifx \keydoc@link \@firstofone
			\PackageWarning{keydoc}{keydoc with optional value but default value is not given}{}%
			\fi
		\fi
		\def\keydoc@keyval{#2}%
		\let\keydoc@valueformat=\keydoc@optionalvalueformat
	\else
		\unless\ifx \keydoc@defaultvalue \pgfkeysnovalue@text
			\PackageWarning{keydoc}{keydoc with default value but value is not optional}{}%
		\fi
		\let\keydoc@valueformat=\keydoc@mandatoryvalueformat
	\fi
}



\newcommand{\keydoc@lookahead}{%
	\futurelet \keydoc@nexttoken \keydoc@lookahead@check
}
\newcommand{\keydoc@lookahead@check}{%
	\ifcat \noexpand\keydoc@nexttoken \space
		\let\keydoc@lookahead@do=\keydoc@lookahead
		\let\keydoc@lookahead@gobble=\@firstofone
	\else\ifx \keydoc@nexttoken \par
		\let\keydoc@lookahead@do=\keydoc@afterlookahead
		\let\keydoc@lookahead@gobble=\@empty
	\else\ifx \keydoc@nexttoken /%
		\keydoc@isrelatedtrue
		\let\keydoc@lookahead@do=\keydoc@lookahead
		\let\keydoc@lookahead@gobble=\@gobble
	\else\ifx \keydoc@nexttoken \keydoc
		\let\keydoc@lookahead@do=\keydoc@afterlookahead
		\let\keydoc@lookahead@gobble=\@empty
	\else\ifx \keydoc@nexttoken \forwardingkeydoc
		\let\keydoc@lookahead@do=\keydoc@afterlookahead
		\let\keydoc@lookahead@gobble=\@empty
	\else\ifx \keydoc@nexttoken \correspondingkeydoc
		\let\keydoc@lookahead@do=\keydoc@afterlookahead
		\let\keydoc@lookahead@gobble=\@empty
	\else\ifx \keydoc@nexttoken \endkeydoc
		\let\keydoc@lookahead@do=\keydoc@afterlookahead
		\let\keydoc@lookahead@gobble=\@empty
	\else
		\ifkeydoc@isrelated
			\PackageError{keydoc}{expexcted another keydoc command after /}{}%
		\fi
		\keydoc@printvalues
		\keydoc@descriptionsep
		\let\keydoc@lookahead@do=\relax
		\let\keydoc@lookahead@gobble=\@empty
	\fi \fi \fi \fi \fi \fi \fi
	\expandafter \keydoc@lookahead@do \keydoc@lookahead@gobble
}
\newcommand{\keydoc@afterlookahead}{%
	\unless\ifkeydoc@isrelated
		\keydoc@printvalues
	\fi
}
\newcommand{\RegisterEnd}[1]{%
	\@tfor \@sec@cmd :=\section\subsection\subsubsection \do{%
		\expandafter\appto\csname My\expandafter\string\@sec@cmd Hook\endcsname{#1}%
	}%
}
\newcommand{\keydoc@registerend}{\RegisterEnd\endkeydoc}

\newcommand{\keylinktarget}[1]{%
	\AbbrevKeyword{key}{\keydoc@keypath/#1} {key}{\keydoc@keypath/\keydoc@keypattern}%
	\AbbrevKeyword{key}{#1} {key}{\keydoc@keypath/\keydoc@keypattern}%
	\AbbrevKeyword{}{#1} {key}{\keydoc@keypath/\keydoc@keypattern}%
	\ignorespaces
}

\let\keydoc@original@key=\key
\newcommand{\keydoc@key@bg}{\formatcode}
% I cannot say \renewcommand{\key}[1][] because that would break \keydoc@original@key
\renewcommand{\key}{%
	\@ifnextchar [
		\key@withopt
		\key@withoutopt
}
\newcommand{\key@withopt}{}
\def\key@withopt[#1]#2{%
	\ifx\relax#1\relax
		\LinkKeyword{key}{#2}\keydoc@key@bg\keydoc@original@key{#2}%
	\else
		\LinkKeyword{key}{#1/#2}\keydoc@key@bg\keydoc@original@key{#2}%
	\fi
}
\newcommand{\key@withoutopt}[1]{%
	\LinkKeyword{key}{#1}\keydoc@key@bg\keydoc@original@key{#1}%
}

\def\key@withopt@unformatted[#1]#2{%
	\ifx\relax#1\relax
		\LinkUnformattedKeyword{key}{#2}{#2}%
	\else
		\LinkUnformattedKeyword{key}{#1/#2}{#2}%
	\fi
}


% ---------- \forwardingkeydoc ---------

\newcommand{\forwardingkeydoc}[2][]{% [#1: name, path or full key of target], #2: name
	\correspondingkeydoc[#1]{#2}{forwarding key}%
}
\newcommand{\correspondingkeydoc}[3][]{% [#1: options], #2: name, #3: key type
	\keydoc[#1, /utils/exec=\forwardingkeydoc@parsetarget{#2}]{#2}{#3}%
}
\newcommand{\forwardingkeydoc@parsetarget}[1]{% #1: this key name
	% parse target which may be a name, path or full key of corresponding key
	\pgfkeysgetvalue{/keydoc/target}{\forwardingkeydoc@targetkey}%
	\ifx\forwardingkeydoc@targetkey\pgfkeysnovalue@text
		\ifundef\TargetKey{%
			\PackageError{keydoc}
				{\string\TargetKey{<keypath>}{<keyname>} must be defined if the the optional argument of \forwardingkeydoc\space is not passed}
				{\string\TargetKey\space shall expand to path/name of the linked key given the path (without trailing slash) and name of the forwarding key.}%
		}{%
			\def\forwardingkeydoc@targetkey{\TargetKey{\keydoc@keypath}}%
			\edef\forwardingkeydoc@targetkey{\expandafter\forwardingkeydoc@targetkey\expandafter{\forwardingkeydoc@stripval#1=\relax}}%
			\expandafter\ifendsonspace\expandafter{\forwardingkeydoc@targetkey}%
				{\def\forwardingkeydoc@targetkey}%
				{\def\forwardingkeydoc@targetkey}%
			\pgfkeyslet{/keydoc/target}{\forwardingkeydoc@targetkey}%
		}%
	\else
		\forwardingkeydoc@parse{\forwardingkeydoc@targetkey}{\keydoc@keypath}{#1}%
		\pgfkeyslet{/keydoc/target}{\forwardingkeydoc@targetkey}%
	\fi
}
\newcommand\forwardingkeydoc@parse[3]{% #1: key name, key path or full key of target, #2: path of source, #3: name of source
	\IfStartsWithSlash{#1}{%
		\edef\forwardingkeydoc@targetkey@original{#1}%
		\edef\forwardingkeydoc@targetkey{#1/\forwardingkeydoc@stripval#3=\relax}%
		\expandafter\ifendsonspace\expandafter{\forwardingkeydoc@targetkey}%
			{\def\forwardingkeydoc@targetkey}%
			{\def\forwardingkeydoc@targetkey}%
		\IfKeywordUndefined{key}{\forwardingkeydoc@targetkey}{%
			\let\forwardingkeydoc@targetkey=\forwardingkeydoc@targetkey@original
		}{}%
	}{%
		\def\forwardingkeydoc@targetkey{#2/#1}%
	}%
}
\def\forwardingkeydoc@stripval#1=#2\relax{#1}

\newcommand{\IfStartsWithSlash}[1]{%
	\expandafter\IfStartsWithSlash@do#1\relax
}
\def\IfStartsWithSlash@do#1#2\relax{%
	\ifx/#1%
		\expandafter\@firstoftwo
	\else
		\expandafter\@secondoftwo
	\fi
}


% ---------- \keytypedoc ---------

\newcommand{\begin@keytypedoc}{\begin{itemize}}
\def        \endkeytypedoc    {\end{itemize}}

\newcommand{\keytypedoc@beforekeytype}{\item}
\newcommand{\keytypedoc@format}[1]{\emph{#1}}
\newcommand{\keytypedoc@descriptionsep}{:}

\newif\if@inkeytypedoc

\newcommand{\keytypeRAW}[1]{#1}
\newcommand{\keytype@bg}[1]{#1}
\newcommand{\keytype}{% [#1: type name], #2: displayed text (usually the type name, but can be another grammatic form)
	\@ifnextchar [
		{\@keytype@do}
		{\@dblarg\@keytype@do}%
}
\def\@keytype@do[#1]#2{%
	\LinkKeyword{keytype}{#1}\keytype@bg\keytypeRAW{#2}%
}

\newcommand{\keytypedoc}[1]{% #1 key type name
	\unless \if@inkeytypedoc
		\begin@keytypedoc
		\keytypedoc@registerend
		\@inkeytypedoctrue
	\fi
	\keytypedoc@beforekeytype[\keytypedoc@keytype{#1}]
	\NewKeyword{keytype}{#1}{\keytype@bg}{\keytypeRAW}{\keytypedoc@format{#1}}%
	\keytypedoc@descriptionsep
}

\newcommand{\keytypedoc@registerend}{\RegisterEnd\endkeytypedoc}

\makeatother
