% !TeX root = ../easyfloats.tex

% \section{Bug reports and contributions}
\section{报告Bug与贡献代码}
\label{bug-reports-and-contributions}

% If you find a bug please open an issue for it
% on \url{https://gitlab.com/erzo/latex-easyfloats/-/issues}
% including a minimal example where the bug occurs,
% an explanation of what you expected to happen
% and the version of \LaTeX\ and the packages you are using (which are included in the log file).
% Issues which are not reproducible will be closed.
如果使用中发现Bug,请在\url{https://gitlab.com/erzo/latex-easyfloats/-/issues}
提交issue。
在提交issue时,强烈建议提交出错时的最小工作示例,
期望结果的说明,
以及\LaTeX{}的版本和使用的宏包(参见log日志文件)。
注意,作者会关闭无法复现的问题。


% If you have a feature request please open an issue for it
% on \url{https://gitlab.com/erzo/latex-easyfloats/-/issues}
% including a minimal example which you would like to work,
% an explanation of what it should do
% and a use case explaining why this would be useful.
如果有新的需求,也请在\url{https://gitlab.com/erzo/latex-easyfloats/-/issues}
中创建一个新的issue进行提交,
在提交issue时,强烈建议提交希望如何工作的最小工作示例,
期望结果的说明,
以及解释为什么这样做会有用的用例。

% Before opening an issue please check that there is not yet an issue for it already.
在创建一个issue之前,请检查该issue是否已经存在。

% If you want to resolve an issue yourself please create a merge request.
% Make the changes in \filename{easyfloats.dtx}.
% You can generate the sty file with \verb|tex easyfloats.ins|
% but you do not need to do that manually because \filename{test/autotest.py} does that automatically for you.
% Before creating a merge request please make sure that the automated tests still pass.
% Run the python3 script \filename{test/autotest.py} from the project root or test directory without arguments.
% While running the tests it shows a progress bar in square brackets.
% A dot stands for a successful test, an F for a failed test and an E for an error in the test script.
% Merge requests where a test prints F will most likely be rejected.
% If you get an E please create a bug report issue.
如果需要解决一个问题,请创建一个PR。
注意,需要在\filename{easyfloats.dtx}中进行修改。
可以在命令行用\verb|tex easyfloats.ins|命令生成 sty 文件,
由于\filename{test/autotest.py}可以自动完成相应操作,实际上并不需要手动操作生成sty文件。
在创建PR之前,请确保通过自动了测试。
可以在项目根目录或test目录下运行无参\filename{test/autotest.py}\ python3脚本,
当测试时,会在方括号内显示一个由方块构成的进度条。
A.表示测试成功,F表示测试失败,
E表示测试脚本中存在错误。
当测试结果为F时,会拒绝合并该PR。
如果得到E,请创建一个issue以报告问题。

% Please use *tabs* for indentation.
请使用*tabs*实现缩进。

% A merge request should include:
一个可合并的PR应该包括:
% \begin{itemize}
% \item The changes to \filename{easyfloats.dtx}
% \item The automatically generated \filename{easyfloats.sty}
% \item Additions to the documentation
% \item Automated tests in the \filename{test} directory to make sure the new feature or bug fix does not break in the future
% \item A link in the merge request description to the issue which it is supposed to close
% \end{itemize}
\begin{itemize}
\item \filename{easyfloats.dtx}文件的变更
\item 自动生成的\filename{easyfloats.sty}文件
\item 添加到说明文档的内容
\item \filename{test}目录中的自动测试,以确保新功能或Bug修复将来不会中断
\item 合并PR描述中指向的应该关闭issue链接
\end{itemize}
