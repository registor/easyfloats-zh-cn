% !TeX root = ../easyfloats.tex

% \section{Used packages}
\section{使用的宏包}
\label{used-packages}

% This package uses the following packages (but depending on the package options it may load more or less packages, see \cref{package-options}):
该宏包使用了以下宏包(注意使用该宏包选项可以选择性使用这些宏包,参见\cref{package-options}):
\pkgdoc{float}
  % for `placement=H` and `float style`.
  为实现`placement=H`和`float style`。
  % It also gives you the possibility to define new float types.
  同时,也可以通过该宏包定义新浮动类型。
\pkgdoc{caption}
  % In the standard document classes there is no distance at all between a table and it's caption above.
  % The caption package fixes this.
  % It also defines the `\phantomcaption` command which I am using in case that no caption is given.
  % (The documentation of `\phantomcaption` is in the \pkg{subcaption} package.)
  % It also gives you the possibility to customize the layout of captions but I am not changing the default layout.
  % And it is a dependency of the \pkg{subcaption} package.
  在标准文件类中,表格与其上方的标题之间完全没有间距。
  caption宏包用于解决此问题。
  当然,它还定义了`\phantomcaption`令,
  如果没有给出标题,将使用该命令。
  (`\phantomcaption`的说明在\pkg{subcaption}宏包文档中)
  它还能够实现字幕的自定义布局,
  但在此没有更改默认布局。
  并且它是\pkg{subcaption}宏包的依赖。
\pkgdoc{subcaption}
  % for \hyperref[subobject-environment]{subobjects}
  用于\hyperref[subobject-environment]{subobjects}
\pkgdoc{graphicx}/\pkgdoc{graphbox}
  % for inserting graphics
  % (see \cmd{\includegraphicobject})
  用于插图
  (参见\cmd{\includegraphicobject})
\pkgdoc{pgfkeys}
  % for parsing key=value lists
  用于解析key=value选项表
\pkgdoc{etoolbox}
  % is a collection of small helpers for programming.
  用于为命令或环境打补丁的工具合集。
\pkgdoc{environ}
  % to define environments which save their content in a macro.
  % I am using this for the `subcaptionbox` backend of the \env{subobject} environment.
  定义将其内容保存在宏中的环境。
  该宏包正在将此用于\env{subobject}环境的`subcaptionbox`后端。
\endpkgdoc
