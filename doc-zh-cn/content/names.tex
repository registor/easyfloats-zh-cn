% !TeX root = ../easyfloats.tex

% \section{Names}
\section{宏包名称}
\label{names}

% You have probably heard the term *floating object* or *float* for short.
% That is mainly what this package is about.
本宏包主要处理\LaTeX{}中浮动体对象或简称浮动体的排版。
% However, I intended to avoid the term *floating* in the name of this package because this package also allows
% to globally disable the floating of those objects.
% Therefore I decided to name this package *objects*.
由于该宏包可全局禁用浮动体,因此,本想尽量避免使用*floating*这样的名称,
所以,一开始将其命名为*objects*。

% This name, however, has been rejected by \TeX\ Live as being too generic.
% And they are right, especially for people with an object oriented programming background that name might be misleading.
% \TeX\ Live has informed me that floating objects are still called floats even if they are technically not floating.
% Therefore I have decided to rename this package to *easyfloats*.
但由于该名称比较泛化,因此,\TeX\ Live拒绝接受这一名称。
当然这是对的,尤其是对那些习惯了面向对象编程的用户,这一名称容易引起误解。
\TeX\ Live在通知中说明,即便从技术上可以禁用浮动体,但它们仍然是浮动体。
因此,最终将该宏包命名为*easyfloats*。

% I have *not* changed the user interface because the package has already been online for more than half a year on my gitlab repository and I don't know how many people are using the package already.
% Therefore all commands and environments defined by this package still carry the old name *object* in them.
由于该宏包已在gitlab上线,并且无法统计有多少用户在使用这一宏包,
因此,虽然命名为*easyfloats*,但对宏包的原接口未做任何发动。
所以,本宏包中所有的命令和环境命名中仍然带有原来的*object*。
